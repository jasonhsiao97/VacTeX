
% Geometry package, setting margin.
\usepackage[top=2.5cm, bottom=2.5cm, left=2.5cm, right=2.5cm]{geometry}
\usepackage{multicol}

% Enum control.
\usepackage{enumerate}

% Hyperlink
\usepackage[%
    pdfstartview=FitH,%
    CJKbookmarks=true,%
    bookmarks=true,%
    bookmarksnumbered=true,%
    bookmarksopen=true,%
    colorlinks=true,%
    citecolor=black,%
    linkcolor=black,%
    anchorcolor=black,%
    urlcolor=black%
]{hyperref}

% Title format
\usepackage{titlesec}
% List of figure, table, etc.
\usepackage{titletoc}
% Table control
\usepackage{booktabs}
% Appendix control
\usepackage[title,titletoc,header]{appendix}

% Font size control
\usepackage{type1cm}
% Indentation control
\usepackage{indentfirst}
\usepackage{changepage}

% Color control
\usepackage{color,xcolor}

%% AMS LaTeX
\usepackage{amsmath,amssymb}
\usepackage{latexsym,textcomp}

%% 数学公式中的黑斜体
% \usepackage{bm}
%% 调整公式字体大小:\mathsmaller, \mathlarger
% \usepackage{relsize}

% graphics
\usepackage{graphicx}
\usepackage{subfig}

%%%% pgf/tikz graph
\usepackage{pgf,tikz}
\usetikzlibrary{shapes,automata,snakes,backgrounds,arrows}
\usetikzlibrary{mindmap}

\usepackage{fancyhdr}
\pagestyle{plain}

%% 有时会出现\headheight too small的warning
\setlength{\headheight}{15pt}


%%%% 设置listings宏包用来粘贴源代码
% Source code
\usepackage{listings}
% Algorithm
\usepackage[ruled,vlined]{algorithm2e}
% Tree graph
\usepackage{synttree}

\lstloadlanguages{}
\lstset{
showstringspaces=false,              %% 设定是否显示代码之间的空格符号
numbers=left,                        %% 在左边显示行号
numberstyle=\tiny,                   %% 设定行号字体的大小
basicstyle=\footnotesize,                    %% 设定字体大小\tiny, \small, \Large等等
keywordstyle=\color{blue!70}, commentstyle=\color{red!50!green!50!blue!50},
                                     %% 关键字高亮
frame=shadowbox,                     %% 给代码加框
rulesepcolor=\color{red!20!green!20!blue!20},
escapechar=`,                        %% 中文逃逸字符,用于中英混排
xleftmargin=2em,xrightmargin=2em, aboveskip=1em,
breaklines,                          %% 这条命令可以让LaTeX自动将长的代码行换行排版
extendedchars=false                  %% 这一条命令可以解决代码跨页时,章节标题,页眉等汉字不显示的问题
}
%%%% listings宏包设置结束

%% 一号, 1.4倍行距
\newcommand{\yihao}{\fontsize{26pt}{36pt}\selectfont}
%% 二号, 1.25倍行距
\newcommand{\erhao}{\fontsize{22pt}{28pt}\selectfont}
%% 小二, 单倍行距
\newcommand{\xiaoer}{\fontsize{18pt}{18pt}\selectfont}
%% 三号, 1.5倍行距
\newcommand{\sanhao}{\fontsize{16pt}{24pt}\selectfont}
%% 小三, 1.5倍行距
\newcommand{\xiaosan}{\fontsize{15pt}{22pt}\selectfont}
%% 四号, 1.5倍行距
\newcommand{\sihao}{\fontsize{14pt}{21pt}\selectfont}
%% 半四, 1.5倍行距
\newcommand{\bansi}{\fontsize{13pt}{19.5pt}\selectfont}
%% 小四, 1.5倍行距
\newcommand{\xiaosi}{\fontsize{12pt}{18pt}\selectfont}
%% 大五, 单倍行距
\newcommand{\dawu}{\fontsize{11pt}{11pt}\selectfont}
%% 五号, 单倍行距
\newcommand{\wuhao}{\fontsize{10.5pt}{10.5pt}\selectfont}

%% 设定段间距
\setlength{\parskip}{0.5\baselineskip}
%% 设定行距
\linespread{1}
%% 中文破折号,据说来自清华模板
\newcommand{\pozhehao}{\kern0.3ex\rule[0.8ex]{2em}{0.1ex}\kern0.3ex}
%% 设定itemize环境item的符号
\renewcommand{\labelitemi}{$\bullet$}

\bibliographystyle{unsrt}
